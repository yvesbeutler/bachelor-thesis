% $Log: abstract.tex,v $
% Revision 1.1  93/05/14  14:56:25  starflt
% Initial revision
% 
% Revision 1.1  90/05/04  10:41:01  lwvanels
% Initial revision
% 
%
%% The text of your abstract and nothing else (other than comments) goes here.
%% It will be single-spaced and the rest of the text that is supposed to go on
%% the abstract page will be generated by the abstractpage environment.  This
%% file should be \input (not \include 'd) from cover.tex.
Nature-Inspired Algorithms können dort eingesetzt werden, wo traditionelle
Problemlösungsmethoden nicht funktionieren, etwa bei Optimierungsproblemen mit einer
viel zu grossen Anzahl an möglichen Lösungen oder bei Problemen mit einer zeitabhängigen
Bewertungsfunktion der Lösungen. Einige dieser Algorithmen haben sich als äusserst performant
und robust erwiesen und werden heute zur Problemlösung in den unterschiedlichsten Anwendungsgebieten
eingesetzt.
Zu der Gruppe der Nature-Inspired Algorithms gehören beispielsweise Evolutionäre-Algorithmen, welche
starke Ähnlichkeiten zu Darwins Evolutionstheorie und seiner These "Survival of the fittest"
aufweisen.

Nebst den Evolutionären Algorithmen werden Schwarm-Algorithmen thematisiert, welche sich an einer
Kollektiven Intelligenz bereichern, um komplexe Problemstellungen zu lösen. Im Tierreich überleben
viele primitive Tierarten nur durch einen Tierverbund. Nach einer kurzen Einführung in die einzelnen Gruppen
wird jede behandelte Unterart der Nature-Inspired Algorithms durch ausgewählte Algorithmen im Detail
veranschaulicht.

Ziel dieser Arbeit ist es, eine Einführung in die Thematik von Nature-Inspired Algorithms und ihren
Einsatzmöglichkeiten zu geben und die bekanntesten Unterarten mit ausgesuchten Algorithmen im Detail
zu erläutern. Nebst den hier thematisierten Evolutionären- und Schwarm-Algorithmen existiert eine
Vielzahl an interessanten Untergruppen wie etwa den Neuronalen- oder den Physikalischen Algorithmen.
